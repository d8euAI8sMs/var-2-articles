\setcounter{chapter}{1}

\section{О выпуклых функциях}

Выпуклой (вверх) называется функция, надграфик которой является выпуклым множеством. Из теории выпуклых множеств следует, что это достигается при выполнении неравенства Йенсена


$$
 f(\alpha x + (1 - \alpha)\ y) \le \alpha f(x) + (1 - \alpha) f(y)
$$


где $0 \le \alpha \le 1$. Если неравенство выполняется для $-f(x)$, то функция называется вогнутой (выпуклой вниз).

Доказываются следующие важные свойства выпуклых функций:

1. Выпуклая функция имеет единственную подозрительную на экстремум точку, причем она будет являться глобальным экстремумом (минимумом для выпуклой вверх и максимумом для выпуклой вниз). Следовательно существует единственная точка $x_0$, для которой верно ${\partial f \over \partial x} \bigg | _{x=x_0} = 0$.

2. Вторая производная сохраняет свой знак ($>0$ для выпуклой и $<0$ для вогнутой функции) и может обратиться в нуль лишь в точке $x_0$.

Аналогичные заключения справедливы и для функции $f(\vec x)$ многих переменных. Условие экстремума тогда можно записать в виде системы


$$
\frac{\partial f(\vec x)}{\partial x_i} = 0 \quad \forall i = 1..n
$$


Возводя каждое уравнение в квадрат и складывая почленно, получим функцию невязки системы


$$
\mathcal E [f(\vec x)] = \sum\limits_{i=1}^n \left(\frac{\partial f(\vec x)}{\partial x_i}\right)^2
$$


Покажем, что функция $\mathcal E$ тоже имеет единственный экстремум. Запишем условие экстремума:


$$
\frac {\partial \mathcal E} {\partial x_j} = \frac {\partial \sum\limits_{i=1}^n \left(\frac{\partial f(\vec x)}{x_i}\right)^2} {\partial x_j} = \sum\limits_{i=1}^n \frac{\partial f(\vec x)}{\partial x_i} \frac{{\partial}^2 f(\vec x)}{\partial x_i \partial x_j} = 0 \quad \forall i = 1..n
$$


Но производные $\frac{\partial f(\vec x)}{\partial x_i}$ обращаются в нуль только при $x=x_0$, а значит в любой другой точке $x=x1\neq x0$ выполнено $\frac{\partial f(\vec x)}{\partial x_i} \neq 0$. Вторые производные $\frac{{\partial}^2 f(\vec x)}{\partial x_i \partial x_j}$ сохраняют свой знак и могут обратиться в нуль лишь при $x=x_0$. Следовательно при $x \neq x_0$ нельзя удовлетворить$\frac {\partial \mathcal E} {\partial x_j} = 0$, что и требовалось доказать.


\section{О выпуклости стабилизирующих функционалов}

Линейные функции вида $f(x) = \mp x$ являются выпуклыми (вогнутыми). Аналогично определяются и линейные функционалы.

То же верно и для квадратичных функций и функционалов. Покажем это для стабилизирующего функционала

$$\Omega[f(x)] = \int\limits_a^b \sum\limits_{n=0}^N q_n(x) \left(\frac{d^{(n)}f(x)}{dx^n}\right)^2 dx$$

Вторая вариация функционала

$${\delta}^2\Omega[f(x)] =
\frac{{\partial}^2 \Omega[f(x) + \alpha\delta f(x)]}{\partial{\alpha}^2} =
\int\limits_a^b \sum\limits_{n=0}^N q_n(x) {\left(\delta f^{(n)}(x)\right)}^2 dx > 0$$

при $q_n(x) > 0 \quad \forall x \in [a, b]$. Следовательно в общем случае выбора правильных $q_n(x)$ стабилизирующий функционал является выпуклым.

Покажем выпуклость функционала $\Omega$ непосредственно по определению выпуклости:

$$\Omega[\alpha f + \beta g] \leq \alpha \Omega[f] + \beta \Omega[g]$$

Действительно, для $n$-го члена суммы в левой части имеем

$$q_n (\alpha f^{(n)} + \beta g^{(n)})^2 = q_n (\alpha^2 (f^{(n)})^2 + \beta^2 (g^{(n)})^2 + 2 \alpha \beta f^{(n)} g^{(n)})$$

а для правой

$$q_n ( \alpha (f^{(n)})^2 + \beta (g^{(n)})^2 )$$

Вычитая одно из другого, получим

$$q_n (\alpha (\alpha - 1) (f^{(n)})^2 + \beta (\beta -1) (g^{(n)})^2 + 2\alpha \beta f^{(n)} g^{(n)})$$

Или, пользуясь тем, что $\beta = 1 - \alpha$,

$$- q_n (\alpha \beta (f^{(n)})^2 + \beta \alpha (g^{(n)})^2 - 2\alpha \beta f^{(n)} g^{(n)}) = - q_n \alpha \beta (f^{(n)} - g^{(n)})^2 < 0 \quad \textrm{iff}\ q_n(x) > 0 \ \forall x \in [a, b]$$

что и требовалось показать.

Выпуклость стабилизирующего функционала $N$-го порядка позволяет использовать его в качестве стабилизатора и в случае метода максимума энтропии, не нарушая выпуклости результирующего функционала.